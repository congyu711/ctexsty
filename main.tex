\documentclass[11pt]{article}

\usepackage[sans]{xenotes}
\usepackage{algo}
\usepackage{metalogo}

\title{Template for \XeLaTeX }
\author{丛宇}
\date{\today}

\usepackage{bm}
\begin{document}
\maketitle
\begin{algo}
sans $\to$ lato + 思源黑体\\
serif $\to$ charter + 思源宋体
\end{algo}

The default math mode font is $Math\ Italic$. This should not be
confused with ordinary \emph{Text Italic} -- notice the different spacing\,!
\verb|\mathbf| produces bold roman letters: $ \mathbf{abcABC} $.
If you wish to embolden complete formulas,
use the \verb|\boldmath| command \emph{before} going into math mode.  
This changes the default math fonts to bold. 
 
\begin{tabular}{ll}
\texttt{normal}   & $ x = 2\pi \Rightarrow x \simeq 6.28 $\\
\texttt{mathbf}   & $\mathbf{x} = 2\pi \Rightarrow \mathbf{x} \simeq 6.28 $\\
\texttt{boldmath} & {\boldmath $x = \mathbf{2}\pi \Rightarrow x 
                   \simeq{\mathbf{6.28}}              $}\\
\end{tabular}
\smallskip

Greek is available in upper and lower case:
$\alpha,\beta \dots \Omega$, and there are special
symbols such as $ \hbar$.
%The following letters should be upright: $\upGamma, \upDelta\dots \upOmega$.
Digits in formulas $1, 2, 3\dots$ may differ from those in text: 1, 2, 3\dots

There is a calligraphic alphabet \verb|\mathcal| for upper case letters
$ \mathcal{ABCDE}\dots $.
%and there are letters for number sets: $\mathbb{A\dots Z} $, which are produced using \verb|\mathbb|.

\noindent
This font has both lining figures (13589, default) and oldstyle figures (\oldstylenums{13589}, select with {\tt$\backslash$oldstylenums\{..\}}). 
%{\em \swshape{I}t \swshape{A}lso \swshape{H}as \swshape{S}wash \swshape{I}talics} {\tt$\backslash$swshape\{..\}}
\\
{\fontseries{c}\selectfont there is also a condensed weight} {\tt$\backslash$fontseries\{c\}$\backslash$selectfont}
 
\begin{equation}
  \phi(t)=\frac{1}{\sqrt{2\pi}}
  \int^t_0 e^{-x^2/2} dx 
\end{equation}

\begin{equation}
  \prod_{j\geq 0}
  \left(\sum_{k\geq 0}a_{jk} z^k\right) 
= \sum_{k\geq 0} z^n
  \left( \sum_{{k_0,k_1,\ldots\geq 0}
          \atop{k_0+k_1+\ldots=n}    }
        a{_0k_0}a_{1k_1}\ldots  \right) 
\end{equation}

\begin{equation}
\pi(n) = \sum_{m=2}^{n}
  \left\lfloor \left(\sum_{k=1}^{m-1}
       \lfloor(m/k)/\lceil m/k\rceil 
       \rfloor \right)^{-1}
  \right\rfloor
\end{equation}

\begin{equation}
\{\underbrace{%
    \overbrace{\mathstrut a,\ldots,a}^{k\ a's},
    \overbrace{\mathstrut b,\ldots,b}^{l\ b's}}
  _{k+1\ \mathrm{elements}}                   \}
\end{equation}

\begin{displaymath}
\mbox{W}^+\
\begin{array}{l}
\nearrow\raise5pt\hbox{$\mu^+ + \nu_{\mu}$}\\
\rightarrow         \pi^+ +\pi^0         \\[5pt]
\rightarrow \kappa^+ +\pi^0              \\
\searrow\lower5pt\hbox{$\mathrm{e}^+ 
          +\nu_{\scriptstyle\mathrm{e}}$}
\end{array}
\end{displaymath}

\begin{displaymath}
\frac{\pm
\left|\begin{array}{ccc}
x_1-x_2  & y_1-y_2 & z_1-z_2 \\
l_1      & m_1     & n_1     \\
l_2      & m_2     & n_2
\end{array}\right|}{
\sqrt{\left|\begin{array}{cc}l_1&m_1\\
l_2&m_2\end{array}\right|^2
+     \left|\begin{array}{cc}m_1&n_1\\
n_1&l_1\end{array}\right|^2
+     \left|\begin{array}{cc}m_2&n_2\\
n_2&l_2\end{array}\right|^2}}
\end{displaymath}

text accents: \`{a},\'{a},\"{a},\^{a}
may differ from math accents:
\begin{displaymath}
\mbox{ acute=}\acute{a}
\mbox{ grave=}\grave{a}
\mbox{ ddot=}\ddot {a}
\mbox{ tilde=}\tilde{a}
\mbox{ bar=}\bar  {a}
\mbox{ breve=}\breve{a}
\mbox{ check=}\check{a}
\mbox{ hat=}\hat  {a}
\mbox{ vec=}\vec  {a}
\mbox{ dot=}\dot  {a}
\end{displaymath}

dotlessi=\i\ 
dotlessj=\j\ 
dagger=$\dagger$\ \ \ 
% \verb|\bm{x}|\ $\bm{x}$ 

%$\hbar$ $\hslash$
\end{document}